\section{Organización del documento}
\label{sec:OrganizacionDocumento}

La presente tesis cuenta con un total de siete capítulos, organizados de la siguiente manera:

El \textit{Capítulo~\ref{chp:MarcoTeorico}} establece el marco teórico, proporcionando los fundamentos de la investigación. En este capítulo, se exploran las bases teóricas y se examinan los antecedentes relevantes que respaldan el estudio.

El \textit{Capítulo~\ref{chp:Metodologia}} describe la metodología utilizada en esta investigación. En este capítulo, se especifica el tipo y diseño de la investigación, la unidad de análisis, la población estudiada y el tamaño de la muestra. Además, se exponen las técnicas empleadas para la recopilación de datos.

El \textit{Capítulo~\ref{chp:Propuesta}} expone la propuesta de investigación, detallando los pasos que se han seguido para la creación, desarrollo y ejecución del modelo de segmentación semántica basado en aprendizaje profundo.

El \textit{Capítulo 5} está dedicado a la exposición, interpretación y discusión de los resultados obtenidos de la investigación.

El \textit{Capítulo 6} presenta las conclusiones de la investigación. En este capítulo, se sintetizan los hallazgos y se responden a los objetivos inicialmente planteados.

El \textit{Capítulo 7} recoge las recomendaciones resultantes de la investigación y se señalan posibles futuras líneas de investigación para ampliar el estudio del tema abordado.