\section{Operacionalización de variables}
\label{sec:OperacionalizacionVariables}

El Cuadro~\ref{tab:OperacionalizacionVariables} se muestra la operacionalización de las variables de investigación.

\begin{landscape}
\begin{table}
\caption{Operacionalización de variables de la investigación}
\footnotesize
\begin{tabularx}{\linewidth}{@{} *5{X} @{}}
\hline
\textbf{Variables} & \textbf{Tipo} & \textbf{Dimensión} & \textbf{Indicadores} \\ \hline
\multirow{11}{=}{\parbox{4cm}{Modelo de segmentación semántica basado en aprendizaje profundo}} & \multirow{11}{=}{\parbox{4cm}{Variable independiente}} & \multirow{4}{=}{\parbox{4cm}{Arquitectura del modelo}} & - Número de capas. \\ &  &  & - Número de parámetros. \\
 &  &  & - Tasa de aprendizaje. \\
 &  &  & - Función de pérdida. \\ \cline{3-4}
 &  & Función de activación & - Función de activación utilizada en las capas ocultas. \\ \cline{3-4}
 &  & Algoritmo de optimización & - Algoritmo utilizado para optimizar el modelo. \\ \cline{3-4}
 &  & \multirow{5}{*}{Conjunto de datos} & - Número de muestras. \\
 &  &  & - Número de muestras utilizadas en el entrenamiento. \\
 &  &  & - Número de muestras utilizadas en la validación. \\
 &  &  & - Número de canales. \\
 &  &  & - Número de clases. \\ \hline
\multirow{8}{=}{\parbox{4cm}{Efectividad del mapeo de glaciares limpios y cubiertos en imágenes satelitales}} & \multirow{8}{=}{\parbox{4cm}{Variable dependiente}} & \multirow{6}{=}{\parbox{4cm}{Precisión}} & - IoU. \\
 &  &  & - Dice. \\
 &  &  & - Accuracy. \\
 &  &  & - Precision. \\
 &  &  & - Recall. \\
 &  &  & - F1 Score. \\ \cline{3-4} 
 &  & Tiempo & - Tiempo requerido para la segmentación semántica en la etapa de pruebas. \\ \cline{3-4}
 &  & Calidad & - Porcentaje de error en la delimitación de glaciares. \\ \hline
\end{tabularx}
\begin{minipage}{\textwidth}
    \vspace{10pt}
    \reference{Elaborado por el autor.}
    \label{tab:OperacionalizacionVariables}
\end{minipage}
\end{table}
\end{landscape}
