\section{Tipo y diseño de la investigación}
\label{sec:TipoDisenoInvestigacion}

\subsection{Tipo de investigación}

La presente investigación se clasifica como cuantitativa y aplicada. La investigación cuantitativa se distingue por el énfasis en la recolección y análisis de datos numéricos, así como su interpretación a través de técnicas estadísticas \cite{kothari2004research, creswell2017research}. Este estudio emplea algoritmos de aprendizaje profundo, los cuales se entrenan y validan con datos numéricos, es decir, los píxeles de las imágenes satelitales y sus correspondientes etiquetas semánticas. Esta estrategia permite un análisis cuantitativo detallado y una evaluación objetiva del rendimiento del modelo de segmentación semántica propuesto.

Además, la investigación se caracteriza como aplicada. A diferencia de la investigación básica, la cual se orienta hacia la comprensión fundamental de los fenómenos, la investigación aplicada tiene como objetivo principal resolver problemas prácticos específicos mediante la aplicación de conocimientos científicos \cite{kothari2004research}. En este caso, el problema que se resuelve es el de mejorar la efectividad del mapeo semiautomático de glaciares limpios y cubiertos en imágenes satelitales.

\subsection{Diseño de la investigación}

En cuanto al diseño de la investigación, el estudio se fundamenta en un diseño experimental. Los diseños experimentales involucran la manipulación de una o más variables independientes para observar su efecto en una variable dependiente, mientras se controlan las demás variables \cite{marczyk2010essentials}. En este estudio, las variables independientes son los parámetros del modelo de segmentación semántica basado en aprendizaje profundo y la variable dependiente es el la efectividad del mapeo de glaciares limpios y cubiertos en imágenes satelitales. Este diseño permite establecer relaciones causales entre las variables y ofrece un mayor control sobre el proceso de investigación.