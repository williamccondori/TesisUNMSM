\section{Técnicas de recolección de datos}
\label{sec:TecnicaRecoleccionDatos}

El Cuadro~\ref{tab:TecnicasInstrumentosRecoleccionDatos} detalla las técnicas e instrumentos de recolección de datos utilizados en la presente investigación:

\begin{table}
\caption{Técnicas e instrumentos de recolección de datos}
\small
\begin{tabularx}{1\textwidth}{XX} 
\hline
\textbf{Técnica} & \textbf{Instrumento} \\ \hline
    \textbf{Documentos y registros}: Se usaron fuentes como documentos oficiales, libros, artículos científicos y estudios de revisiones sistemáticas. Esto con el objetivo de respaldar el marco teórico y aclarar los conceptos importantes para la investigación actual. Esta recolección de datos se realizó mediante una metodología basada en criterios de selección definidos. & \textbf{Bases de datos bibliográficas}: Se emplearon múltiples bases de datos bibliográficas para realizar la recopilación de antecedentes de la investigación y de trabajos relacionados, tal y como se especifica en \ref{suc:RevisionSistematicaLiteratura}. \newline \textbf{Bibliotecas digitales}: Se hizo uso de múltiples bibliotecas digitales para la recopilación de libros, artículos e informes relacionados a la presente investigación. \newline  \textbf{Archivos institucionales}: Se obtuvo información de archivos institucionales, principalmente de informes de inventarios glaciares de países de la región. \newline  \textbf{Gestores de referencias}: Si hizo uso de gestores de referencias como \href{https://www.zotero.org/}{Zotero} y \href{https://www.mendeley.com/}{Mendeley}. De igual manera, se emplearon herramientas de gestión para efectuar la revisión sistemática de la literatura, tal como \href{https://parsif.al/}{Parsifal}.\\ \hline
\textbf{Técnicas digitales}: Se recopiló información por medio de herramientas en línea. Esta información incluyó archivos, imágenes, vídeos, conjuntos de datos, información vectorial, imágenes satelitales, modelos digitales de elevación, y datos históricos esquematizados. & \textbf{Fichas de descarga}: Toda la información primaria adquirida a través de internet se documentó en una ficha de descarga. Esta ficha incluyó detalles como la fuente de la información, el contenido del archivo y la fecha de adquisición. \\ \hline
\textbf{Experimentos}: Se extrajo información de los experimentos ejecutados, los cuales se fundamentaron en métricas predefinidas. Este proceso implicó la adquisición de estas métricas, documentación de resultados y el seguimiento de hiperparámetros durante las etapas de entrenamiento, evaluación y pruebas del modelo. & \textbf{Registro de pruebas}: Se utilizó un registro de pruebas que documentó los hiperparámetros establecidos para cada modelo, junto con los resultados adquiridos en las fases de entrenamiento, evaluación y pruebas del modelo. \newline \textbf{Herramientas de gestión de experimentos}: Se utilizó una herramienta digital llamada \href{https://wandb.ai/}{Wandb} para la gestión de experimentos. Esta permitió organizar las ejecuciones, preservar los resultados obtenidos y los hiperparámetros establecidos. Cabe destacar que esta herramienta fue compatible únicamente con modelos programados en Python, a diferencia del registro de pruebas.
\\  \hline
\end{tabularx}
\begin{minipage}{\textwidth}
    \vspace{10pt}
    \reference{Elaborado por el autor.}
    \label{tab:TecnicasInstrumentosRecoleccionDatos}
\end{minipage}
\end{table}